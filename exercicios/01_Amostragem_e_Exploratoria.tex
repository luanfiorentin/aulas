\documentclass[a4paper,11pt,fleqn]{article}\usepackage[]{graphicx}\usepackage[]{color}
%% maxwidth is the original width if it is less than linewidth
%% otherwise use linewidth (to make sure the graphics do not exceed the margin)
\makeatletter
\def\maxwidth{ %
  \ifdim\Gin@nat@width>\linewidth
    \linewidth
  \else
    \Gin@nat@width
  \fi
}
\makeatother

\definecolor{fgcolor}{rgb}{0, 0, 0}
\newcommand{\hlnum}[1]{\textcolor[rgb]{0,0,0}{#1}}%
\newcommand{\hlstr}[1]{\textcolor[rgb]{0,0,0}{#1}}%
\newcommand{\hlcom}[1]{\textcolor[rgb]{0.4,0.4,0.4}{\textit{#1}}}%
\newcommand{\hlopt}[1]{\textcolor[rgb]{0,0,0}{\textbf{#1}}}%
\newcommand{\hlstd}[1]{\textcolor[rgb]{0,0,0}{#1}}%
\newcommand{\hlkwa}[1]{\textcolor[rgb]{0,0,0}{\textbf{#1}}}%
\newcommand{\hlkwb}[1]{\textcolor[rgb]{0,0,0}{\textbf{#1}}}%
\newcommand{\hlkwc}[1]{\textcolor[rgb]{0,0,0}{\textbf{#1}}}%
\newcommand{\hlkwd}[1]{\textcolor[rgb]{0,0,0}{\textbf{#1}}}%

\usepackage{framed}
\makeatletter
\newenvironment{kframe}{%
 \def\at@end@of@kframe{}%
 \ifinner\ifhmode%
  \def\at@end@of@kframe{\end{minipage}}%
  \begin{minipage}{\columnwidth}%
 \fi\fi%
 \def\FrameCommand##1{\hskip\@totalleftmargin \hskip-\fboxsep
 \colorbox{shadecolor}{##1}\hskip-\fboxsep
     % There is no \\@totalrightmargin, so:
     \hskip-\linewidth \hskip-\@totalleftmargin \hskip\columnwidth}%
 \MakeFramed {\advance\hsize-\width
   \@totalleftmargin\z@ \linewidth\hsize
   \@setminipage}}%
 {\par\unskip\endMakeFramed%
 \at@end@of@kframe}
\makeatother

\definecolor{shadecolor}{rgb}{.97, .97, .97}
\definecolor{messagecolor}{rgb}{0, 0, 0}
\definecolor{warningcolor}{rgb}{1, 0, 1}
\definecolor{errorcolor}{rgb}{1, 0, 0}
\newenvironment{knitrout}{}{} % an empty environment to be redefined in TeX

\usepackage{alltt}

%%----------------------------------------------------------------------
%% opções comuns
\usepackage[brazilian]{babel}
\usepackage[utf8]{inputenc}
\usepackage[T1]{fontenc}
\usepackage{textcomp}
%\usepackage[margin=2cm]{geometry}
\usepackage{indentfirst}
\usepackage{fancybox}
%\usepackage[usenames,dvipsnames]{color}
\usepackage{amsmath,amsfonts,amssymb,amsthm}
\usepackage{lscape}
\usepackage{natbib}
\setlength{\bibsep}{0.0pt}
\usepackage{url}
\usepackage{multicol}
\usepackage{multirow}
\usepackage[final]{pdfpages}
\usepackage{setspace}
\usepackage{paralist} % enumitem, compactitem
%%----------------------------------------------------------------------

%%----------------------------------------------------------------------
%% FLOATS: graficos e tabelas
\usepackage{graphicx}
\usepackage{float} % fornece a opção [H] para floats
\usepackage{longtable}
\usepackage{supertabular}
%% captions e headings em sans-serif
\usepackage[font={sf},labelfont={sf,bf}]{caption}
\usepackage{subcaption}
\renewcommand{\thesubfigure}{\Alph{subfigure}}
\usepackage{titlesec}
\titleformat*{\section}{\normalsize\bfseries\sffamily}
\titleformat*{\subsection}{\normalsize\bfseries\sffamily}
\titleformat*{\subsubsection}{\normalsize\bfseries\sffamily}
\titleformat*{\paragraph}{\normalsize\bfseries\sffamily}
\titleformat*{\subparagraph}{\normalsize\bfseries\sffamily}
\theoremstyle{definition}
\newtheorem*{mydef}{Definição}
%%----------------------------------------------------------------------

%%----------------------------------------------------------------------
%% definiçoes de hyperref e xcolor
\usepackage{hyperref}
\usepackage{xcolor}
%%----------------------------------------------------------------------

%%----------------------------------------------------------------------
%% FONTES

%% micro-tipografia
\usepackage[protrusion=true,expansion=true]{microtype}
%% Bitstream Charter with mathdesign
\usepackage{lmodern} % sans-serif: Latin Modern
\usepackage[charter]{mathdesign} % serif: Bitstream Charter
\usepackage[scaled]{beramono} % truetype: Bistream Vera Sans Mono
\usepackage[scaled]{helvet}
%\usepackage{inconsolata}


%\usepackage[sf]{titlesec}
%%----------------------------------------------------------------------

%%----------------------------------------------------------------------
%% hifenização
\usepackage[htt]{hyphenat} % permite hifenizar texttt. Ao inves disso
% pode usar \allowbreak no ponto qu quiser quebrar dentro do texttt
\hyphenation{con-si-de-ra-ção pes-que-i-ros pes-que-i-ra se-gui-do-ras
  di-fe-ren-tes pla-ni-lha pla-ni-lhão re-fe-ren-te con-ta-gem
  em-bar-ques qua-li-da-de a-le-a-to-ri-za-dos}
%%----------------------------------------------------------------------

%%----------------------------------------------------------------------
%% comandos customizados
\usepackage{xspace} % lida com os espaços depois dos comandos
\providecommand{\eg}{\textit{e.g.}\xspace}
\providecommand{\ie}{\textit{i.e.}\xspace}
\providecommand{\R}{\textsf{R}\xspace}
\newcommand{\mb}[1]{\mathbf{#1}}
\newcommand{\bs}[1]{\boldsymbol{#1}}
\providecommand{\E}{\text{E}}
\providecommand{\Var}{\text{Var}}
\providecommand{\logit}{\text{logit}}
%% Para alterar o titulo do thebibliography
\addto\captionsbrazilian{%
  \renewcommand{\refname}{Bibliografia}
}
%%----------------------------------------------------------------------

%%----------------------------------------------------------------------
%% Comandos para deixar o texto mais compacto
\usepackage{marginnote}
\usepackage[top=1cm, bottom=1cm, inner=1cm, outer=1cm,nohead, nofoot, heightrounded, marginparsep=.05cm]{geometry}
\setlength{\parindent}{0pt}
%%----------------------------------------------------------------------
\IfFileExists{upquote.sty}{\usepackage{upquote}}{}
\begin{document}

\reversemarginpar % para colocar a marginnote a esquerda





\hrule
\vspace{0.3cm}

\begin{minipage}[c]{.85\textwidth}
  Estatística II --- CE003 \\
  Prof. Fernando de Pol Mayer --- Departamento de Estatística - DEST \\
  Exercícios: amostragem e análise exploratória de dados \\
  Nome:  \hfill GRR: \hspace{2cm}
\end{minipage}\hfill
\begin{minipage}[c]{.15\textwidth}
\flushright
\includegraphics[width=2.2cm]{../img/ufpr-logo.png}
\end{minipage}

\vspace{0.3cm}
\hrule
\vspace{0.3cm}
%%----------------------------------------------------------------------

\begin{compactenum}[1.]
\item Defina, com suas palavras, o que é população e o que é
  amostra. Faça a distinção entre parâmetro populacional e estatística
  amostral. Quais são as duas condições mínimas para que uma amostra
  possa ser considerada representativa da população?
\end{compactenum}

\vspace{0.3cm}
\hrule
\vspace{0.3cm}

\begin{compactenum}[2.]
\item Nos itens abaixo indique se na situação temos
  uma estatística (E) ou um parâmetro (P).
  \begin{compactenum}
\item Tem-se interesse na altura média dos jogadores de basquete
  profissional no Brasil. São medidas as alturas de 30\% dos jogadores
  profissionais dos diversos times nacionais, e então a média é
  calculada.
\item O objetivo é obter a biomassa (peso total) de determinada espécie
  de planta em uma floresta. São pesadas todas as folhas caídas desta
  espécie, coletados em conglomerados amostrais com 1 m$^2$ cada um
  distribuídos aleatoriamente de forma a cobrir 1\% da área da
  floresta.
\item Tem-se interesse no comprimento médio dos peixes criados em
  um tanque de cultivo. Na despesca o tanque é esgotado, os
  peixes são abatidos e medidos, e finalmente se calcula uma média.
% \item Tem-se interesse no número de raios da $1^a$ nadadeira
%   dorsal do estoque de ``cavalinha'' (pequeno peixe pelágico)
%   que habita águas da região sudoeste do Atlântico Sul. Um barco faz
%   um lance de pesca em um dos cardumes avistados e o           número
%   de raios de todos os peixes capturados é contado.
\item Foi perguntado para 100 estudantes, selecionados aleatoriamente, o
  tempo médio de espera na fila do RU. O resultado foi de 20 minutos.
\item Em um estudo de todos os 2223 passageiros do Titanic, verificou-se
  que 706 sobreviveram quando ele afundou.
\item Há interesse em se estudar a altura de uma espécie de árvore
  e após medir 10 delas entre várias presentes no local da
  coleta calculamos que em média uma árvore mede 4,14 metros.
\item O Senado atual de um país é composto por 87 homens e 13 mulheres.
  \end{compactenum}
\end{compactenum}

\vspace{0.3cm}
\hrule
\vspace{0.3cm}

\begin{compactenum}[3.]
\item Indique que tipo de amostragem (aleatória simples, sistemática,
  estratificada ou conglomerado) está sendo
  utilizada em cada um dos casos descritos abaixo.
  \begin{compactenum}
  \item Um bairro com 1000 quadras é dividido em 1000 parcelas com 1
    quadra cada. São sorteadas 40 parcelas e todos os moradores (maiores
    de idade) são questionados sobre a eficácia das obras da prefeitura
    naquele bairro.
  \item Em uma área arborizada são escolhidas dez árvores ao acaso, e os
    diâmetros delas são então medidos.
  \item Em uma indústria de mineração, sedimentos são transportados
    sobre uma esteira a uma determinada velocidade. Um
    pesquisador parado ao lado da esteira interrompe seu andamento e
    coleta uma amostra de 5 em 5 minutos.
  \item Interessado em estudar os pesos de uma espécie de ave, um
    pesquisador seleciona ao acaso 30 fêmeas e também 30 machos.
  \item Uma Universidade fez um estudo sobre o hábito de bebida dos
    estudantes, selecionando aleatoriamente 10 classes diferentes, e
    entrevistando todos os alunos de cada uma das classes.
  \item Um Economista está estudando o efeito da educação sobre o
    salário, e realiza uma pesquisa com 150 trabalhadores selecionados
    aleatoriamente de cada uma das seguintes categorias: menos do que o
    ensino médio, ensino médio, mais do que o ensino médio.
  % \item Uma área de manguezal é subdividida em 200 parcelas, dentro de
  %   cada parcela são sorteadas 2 árvores, dessas árvores são sorteadas
  %   uma quantidade de galhos, os quais são então medidos.
  \end{compactenum}
\end{compactenum}

\vspace{0.3cm}
\hrule
\vspace{0.3cm}

\begin{compactenum}[4.]
\item Identifique o tipo de cada uma das variáveis aleatórias descritas
  abaixo como: Qualitativa nominal (N), Qualitativa ordinal
  (O), Quantitativa discreta (D) e Quantitativa contínua (C).
  \begin{compactenum}
   \item Medidas de diâmetro (mm) de um rolamento industrial.
   \item Número de ações da Petrobrás disponíveis no mercado financeiro
     em 2013.
   \item Nomes de empresas de Curitiba com mais de 100 funcionários.
   \item Número de estudantes que frequentam o RU diariamente
   \item Temperatura das câmaras frias de uma indústria de pescado.
   \item Peso de algas coletadas em um determinado banco submerso.
   \item Avaliação em graus de qualidade (``bom'', ``razoável'',
     ``ruim'') de um equipamento industrial
   \item Tempo que você dispenderá estudando para a prova de
     estatística.
   \end{compactenum}
\end{compactenum}

\vspace{0.3cm}
\hrule
\vspace{0.3cm}

\clearpage

\vspace{0.3cm}
\hrule
\vspace{0.3cm}

\begin{compactenum}[5.]
\item Um professor pergunta a cada um de seus alunos que ramo do
  conhecimento prefere estudar: Línguas e Literatura (L\&L), Ciências
  Exatas (CE), Ciências Físicas e Naturais (F\&N), Artes e Música
  (A\&M). Organize a distribuição das frequências com as respostas
  abaixo e construa um gráfico de barras.
  \begin{table}[!h]
    \centering
    \begin{tabular}{ccccc}
      \hline
      A\&M & L\&L & F\&N & A\&M & F\&N \\
      L\&L & L\&L & L\&L & L\&L & A\&M\\
      F\&N & A\&M & CE & F\&N & \\
      CE & F\&N & F\&N & A\&M & \\
      CE & F\&N & CE & L\&L \\
      \hline
    \end{tabular}
  \end{table}
\end{compactenum}

\vspace{0.3cm}
\hrule
\vspace{0.3cm}

\begin{compactenum}[6.]
\item A tabela de distribuições de frequência abaixo apresenta o tempo
  (em minutos) que uma pessoa leva para encontrar um livro na estante
  de uma biblioteca, após consultar o sistema e saber o número de
  referência do livro.
  \begin{table}[h]
    \centering
    \begin{tabular}{cr}
      \hline
      \textbf{Classes} & \textbf{Frequência} \\
      \hline
      $0,5 \vdash 1,0$ & 1 \\
      $1,0 \vdash 1,5$ & 3 \\
      $1,5 \vdash 2,0$ & 22 \\
      $2,0 \vdash 2,5$ & 115 \\
      $2,5 \vdash 3,0$ & 263 \\
      $3,0 \vdash 3,5$ & 287 \\
      $3,5 \vdash 4,0$ & 99 \\
      $4,0 \vdash 4,5$ & 32 \\
      \hline
    \end{tabular}
  \end{table}
  \begin{compactenum}
  \item Complete a tabela com as frequências: relativa, acumulada, e
    relativa acumulada
  \item Em quantos minutos (intervalo de minutos) é mais comum as
    pessoas encontrarem o livro?
  \item Qual a porcentagem de pessoas que levam menos de 3 minutos para
    encontrar o livro?
  \item Quantas pessoas levam menos de 2 minutos para encontrar o livro?
  \item Qual o percentual de pessoas que levam entre 4 minutos
    (inclusive) e 4,5 minutos?
  \item Qual o percentual de pessoas que levam no máximo 1 minuto para
    encontrar o livro?
  \end{compactenum}
\end{compactenum}

\vspace{0.3cm}
\hrule
\vspace{0.3cm}

\begin{compactenum}[7.]
\item Construa uma tabela com as distribuições de frequência absoluta,
  relativa, absoluta acumulada e relativa acumulada usando a amostra do
  número de páginas de livros infanto-juvenis dada por
\begin{verbatim}
46 46 53 30 62 50 69 49 58 65
62 52 44 38 33 60 50 39 53 50
64 53 45 38 31 41 56 54 38 42
31 38 66 29 41 55 43 50 40 45
\end{verbatim}
  Construa um histograma com a densidade de frequência, e um gráfico de
  ramo-e-folhas. O que você pode interpretar destes dados a partir da
  tabela e dos gráficos?
\end{compactenum}

\vspace{0.3cm}
\hrule
\vspace{0.3cm}

\begin{compactenum}[8.]
\item Qual a diferença entre um gráfico de barras e um histograma?
\end{compactenum}

\vspace{0.3cm}
\hrule
\vspace{0.3cm}

\end{document}
